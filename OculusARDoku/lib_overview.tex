\section{ARLib}\label{arlib}

This is the main library we are developing here. It is comprised of the
following modules:

\begin{itemize}
\item
  \hyperref[sec:video_library]{ARLibVideo}:
  The stereo video input library. It grabs real-time video streams from
  the two cameras attached to the Oculus Rift DK2 and provides
  coordinate lookup tables to undistort them according to their lens
  parameters and align them to one another according to their
  precalculated homographies.
\item
  \hyperref[sec:oculus_library]{ARLibOculus}:
  A small wrapper for LibOVR. Only the needed parts. It initializes the
  HMD, provides the LibOVR configuration parameters and the latest
  LibOVR tracking data.
\item
  \hyperref[sec:tracking_library]{ARLibTracking}:
  The rigid body tracking library. It uses the NatNetSDK and/or
  ARLibOculus to provide the latest positions and orientations of rigid
  bodies as well as a fraction of their tracking history for retroactive
  queries.
\item
  \hyperref[sec:ogre_library]{ARLibOgre}:
  The OGRE 1.9 interface library. The only part of ARLib that depends on
  OGRE! Provides all the shaders, materials, render targets and
  abstracted components that can be directly inserted into a scene.
\end{itemize}

\section{ARLib\_Samples}\label{arlibux5fsamples}

We provide two simple example projects that demonstrate the usage of
ARLib within OGRE:

\begin{itemize}
\item
  \hyperref[sec:simple_scene]{SimpleScene}:
  A simple example scene with some static objects placed around the
  origin.
\item
  \hyperref[sec:remote_game]{RemoteGame}:
  A simple game where one has to defend against laser bullets of a Star
  Wars Remote.
\end{itemize}