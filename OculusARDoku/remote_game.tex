\begin{center}
\textbf{Developed by Florian Kleene, Christian Thiele \& Andreas Mantler}
\end{center}

\includegraphics*[width=\textwidth]{RemoteGameThumb.png}

\emph{Remote Game} is a more advanced example. In addition to the \hyperref[simple_scene]{SimpleScene}, this example includes some physics and gameplay code. Additionally it showcases interaction between the user and dynamic virtual objects. Games and other augmented reality applications can easily be developed in a similar fashion to this project, which is quite similar to the way it is done in OgreBullet.

In this simplistic game, the player is equipped with a lightsaber, which is controlled via the tracking library, and has to defend himself against a jedi training drone that is continuously shooting laser beams at the player. The goal is to shoot the laser beams back at the drone. 

A standard calibration step is necessary before the game can be played much like in the \textit{Simple Scene} sample. To do this the player has to stand in an upright position and hold the rigid body, that represents the lightsaber, pointing upwards aswell. 



\section{Keybindings}\label{keybindings}

\begin{itemize}
\itemsep1pt\parskip0pt\parsep0pt
\item
  \textbf{`V'}: Pull out the lightsaber.
\item
  \textbf{`C'}: Set the origin of the scene to the current position and
  orientation of the Oculus Rift.
\item
  \textbf{`N'}: Switch between the Glow Compositor,
  \hyperref[sec:postprocessing]{Watercolor
  postprocessing effect}, both and none of them.
\item
  \textbf{`9'}: Increase the amount of constant video latency (for the
  retroactive positioning of video frames in the scene).
\item
  \textbf{`0'}: Decrease the amount of constant video latency (for the
  retroactive positioning of video frames in the scene).
\item
  \textbf{`Q'}: Increase the interpupillary distance in the video
  background.
\item
  \textbf{`E'}: Decrease the interpupillary distance in the video
  background.
\item
  \textbf{`CTRL'}: Eye selection. The left eye is selected by default.
  While the key is pressed, the right eye is selected.
\item
  \textbf{`D'}: Move the video placement for the selected eye to the
  right.
\item
  \textbf{`A'}: Move the video placement for the selected eye to the
  left.
\item
  \textbf{`W'}: Move the video placement for the selected eye up.
\item
  \textbf{`S'}: Move the video placement for the selected eye down.
\item
  \textbf{RIGHT}: Increase the size of the selected eye horizontally.
\item
  \textbf{LEFT}: Decrease the size of the selected eye horizontally.
\item
  \textbf{UP}: Increase the size of the selected eye vertically.
\item
  \textbf{DOWN}: Decrease the size of the selected eye vertically.
\end{itemize}