\begin{center}
\textbf{Developed by Andreas Mantler}
\end{center}
\includegraphics*[width=\textwidth]{SimpleSceneThumb.png}

This is the basic example. It shows the initialization and usage of ARLib for a simple Augmented Reality task in OGRE. This example can also be used to test the hardware setup. Please note that none of the hardware components have to be connected to start this example. It will tell you in the console/logfile if there were any detected problems. To keep the example simple, only three static cubes are added to the Scene. Using the tracking system, you should be able to look and walk around them and see how stable their appearance is. Using the keybindings below, you may want to finetune the video positions and sizes as well as their constant hardware latency for a smooth experience.

\section{Keybindings}\label{keybindings-1}

\begin{itemize}
\itemsep1pt\parskip0pt\parsep0pt
\item
  \textbf{`C'}: Set the origin of the scene to the current position and
  orientation of the Oculus Rift.
\item
  \textbf{`N'}: Toggle the
  \hyperref[sec:postprocessing]{Non-Photorealistic
  Rendering watercolor postprocessing effect} on/off.
\item
  \textbf{`9'}: Increase the amount of constant video latency (for the
  retroactive positioning of video frames in the scene).
\item
  \textbf{`0'}: Decrease the amount of constant video latency (for the
  retroactive positioning of video frames in the scene).
\item
  \textbf{`Q'}: Increase the interpupillary distance in the video
  background.
\item
  \textbf{`E'}: Decrease the interpupillary distance in the video
  background.
\item
  \textbf{`CTRL'}: Eye selection. The left eye is selected by default.
  While the key is pressed, the right eye is selected.
\item
  \textbf{`D'}: Move the video placement for the selected eye to the
  right.
\item
  \textbf{`A'}: Move the video placement for the selected eye to the
  left.
\item
  \textbf{`W'}: Move the video placement for the selected eye up.
\item
  \textbf{`S'}: Move the video placement for the selected eye down.
\item
  \textbf{RIGHT}: Increase the size of the selected eye horizontally.
\item
  \textbf{LEFT}: Decrease the size of the selected eye horizontally.
\item
  \textbf{UP}: Increase the size of the selected eye vertically.
\item
  \textbf{DOWN}: Decrease the size of the selected eye vertically.
\end{itemize}