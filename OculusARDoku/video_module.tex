In order to grab the video from the two Logitech C310 webcams we used the capture
class from Microsofts Media Foundation example
\href{https://msdn.microsoft.com/en-us/library/windows/desktop/ee663604\%28v=vs.85\%29.aspx}{MFCaptureToFileSample}
and modified
\href{https://github.com/ands/OculusMeetsAR/blob/master/ARLib/src/Video/CCapture.cpp}{it}.
If you want to use different cameras, video resolution or drivers,
you'll have to modify this class (especially the ConfigureSourceReader
function, where we set the video format). Furthermore you'll need to
modify the camera selection in the constructor of
\href{https://github.com/ands/OculusMeetsAR/blob/master/ARLib/src/Video/videoplayer.cpp}{VideoPlayer.cpp}.

However if you just want to use the video module with two C310 and the
drivers provided by Logitech, you'll just have to create a VideoPlayer
instance via

\texttt{VideoPlayer(int\ cameraNumber\ =\ 0,\ const\ char\ *ocamModelParametersFilename\ =\ NULL,\ float\ videoDistance\ =\ 4.0f,\ const\ char\ *homographyMatrixFilename\ =\ NULL);}

In order to undistort the video (using the ocam parameters mentioned in
the \hyperref[sec:stereo_calibration]{calibration section}) and rectify it according to a previously approximated homography, you need to hand over the filenames. Furthermore we used cameraNumber 0 for the left and 1 for the right camera. The videoDistance is similar to a zoom factor applied to the video (look at \texttt{void\ calculateUndistortionMap(float\ *xyMap)} to see its effect).

After initialization the videoPlayer will immediately start capturing. To get a pointer to the current video frame memory in the BGR format, you then just have to call videoPlayer's function
\texttt{void\ *\ update(LARGE\_INTEGER\ *captureTimeStamp\ =\ NULL);}.

\section{Tips \& Problem solving}\label{tips-problem-solving}

\begin{itemize}
\itemsep1pt\parskip0pt\parsep0pt
\item
  One camera can be attached to the built-in USB port of the Oculus Rift DK2, if the DK2s additional power supply is used.
\item
  We couldn't get more than one video stream over that built-in USB port. The second camera needs a separate USB cord.
\item
  Sometimes, the Logitech driver enables a digital zoom (Probably to lower the amount of data transferred over a USB root hub). Use a different USB port for one of the cameras.
\item
  Camera enumeration seems to happen in exactly the same order each time.
\end{itemize}