The aim of this project is to equip the Oculus Rift DK2 with a stereo camera system and build a framework in order to extend its functionality regarding AR (augmented reality). While AR applications using smartphones or wearables like Google Glass are only able to augment a small part of our FOV (field of view), head mounted displays (HMD) like Oculus Rift are not limited in this regard and are therefore creating a more immersive experience. Furthermore the use of a HMD enables the user to switch between VR (virtual reality) and AR.\\
In addition we integrate a tracking system, which enables us to get the exact position of the HMD as well as the position of any other object equipped with so-called rigid bodies. This allows interactions with the virtual objects supplemented to the real-world environment without using further input devices.

\section*{Previous Work}
There are mainly two similar (published) projects to ours. The first one is the commercial \href{http://ovrvision.com/}{OVRVision} by Shinobiya.com Co.Ltd., which is available for Oculus Rift DK2. The OVRVision has two fixed, parallel mounted cameras and therefore lacks the option to adjust the position of these cameras to match the user's IPD (interpupillary distance). Secondly there is William Steptoe's well documented project \href{http://willsteptoe.com/post/66968953089/ar-rift-part-1}{AR-Rift}, which is more flexible regarding the cameras' positions, but is based on the Oculus Rift DK1, which has a much lower resolution than the DK2 used for our project. Nevertheless our project is built upon ideas of both these AR-extensions for Oculus Rift.